%
% This is an example LaTeX file which uses the SANDreport class file.
% It shows how a SAND report should be formatted, what sections and
% elements it should contain, and how to use the SANDreport class.
% It uses the LaTeX article class, but not the strict option.
% ItINLreport uses .eps logos and files to show how pdflatex can be used
%
% Get the latest version of the class file and more at
%    http://www.cs.sandia.gov/~rolf/SANDreport
%
% This file and the SANDreport.cls file are based on information
% contained in "Guide to Preparing {SAND} Reports", Sand98-0730, edited
% by Tamara K. Locke, and the newer "Guide to Preparing SAND Reports and
% Other Communication Products", SAND2002-2068P.
% Please send corrections and suggestions for improvements to
% Rolf Riesen, Org. 9223, MS 1110, rolf@cs.sandia.gov
%
\documentclass[pdf,12pt]{report}
% pslatex is really old (1994).  It attempts to merge the times and mathptm packages.
% My opinion is that it produces a really bad looking math font.  So why are we using it?
% If you just want to change the text font, you should just \usepackage{times}.
% \usepackage{pslatex}
\usepackage{times}
\usepackage[FIGBOTCAP,normal,bf,tight]{subfigure}
\usepackage{amsmath}
\usepackage{amssymb}
\usepackage{pifont}
\usepackage{enumerate}
\usepackage{listings, color}
\definecolor{forestgreen}{RGB}{34,139,34}
\definecolor{orangered}{RGB}{239,134,64}
\definecolor{darkblue}{rgb}{0.0,0.0,0.6}
\definecolor{gray}{rgb}{0.4,0.4,0.4}
\setcounter{secnumdepth}{5}
\lstdefinestyle{XML} {
    language=XML,
    extendedchars=true,
    breaklines=true,
    breakatwhitespace=true,
    emph={name,dim,interactive,overwrite},
    emphstyle=\color{red},
    basicstyle=\ttfamily,
    columns=fullflexible,
    commentstyle=\color{gray}\upshape,
    morestring=[b]",
    morecomment=[s]{<?}{?>},
    morecomment=[s][\color{forestgreen}]{<!--}{-->},
    keywordstyle=\color{cyan},
    stringstyle=\ttfamily\color{black}\normalfont,
    tagstyle=\color{darkblue}\bf,
    morekeywords={attribute,source,variables,version,type,release,x,z,y,xlabel,ylabel,how,text,param1,param2,color,label},
}


%%%%%%%% Begin comands definition to input python code into document
\usepackage[utf8]{inputenc}

% Default fixed font does not support bold face
\DeclareFixedFont{\ttb}{T1}{txtt}{bx}{n}{9} % for bold
\DeclareFixedFont{\ttm}{T1}{txtt}{m}{n}{9}  % for normal

% Custom colors
\usepackage{color}
\definecolor{deepblue}{rgb}{0,0,0.5}
\definecolor{deepred}{rgb}{0.6,0,0}
\definecolor{deepgreen}{rgb}{0,0.5,0}

\usepackage{listings}

% Python style for highlighting
\newcommand\pythonstyle{\lstset{
language=Python,
basicstyle=\ttm,
otherkeywords={self, none, return},             % Add keywords here
keywordstyle=\ttb\color{deepblue},
emph={MyClass,__init__},          % Custom highlighting
emphstyle=\ttb\color{deepred},    % Custom highlighting style
stringstyle=\color{deepgreen},
frame=tb,                         % Any extra options here
showstringspaces=false            %
}}


% Python environment
\lstnewenvironment{python}[1][]
{
\pythonstyle
\lstset{#1}
}
{}

% Python for external files
\newcommand\pythonexternal[2][]{{
\pythonstyle
\lstinputlisting[#1]{#2}}}

% Python for inline
\newcommand\pythoninline[1]{{\pythonstyle\lstinline!#1!}}
%%%%%%%% End comands definition to input python code into document

%\usepackage[dvips,light,first,bottomafter]{draftcopy}
%\draftcopyName{Sample, contains no OUO}{70}
%\draftcopyName{Draft}{300}

% The bm package provides \bm for bold math fonts.  Apparently
% \boldsymbol, which I used to always use, is now considered
% obsolete.  Also, \boldsymbol doesn't even seem to work with
% the fonts used in this particular document...
\usepackage{bm}

% Define tensors to be in bold math font.
\newcommand{\tensor}[1]{{\bm{#1}}}

% Override the formatting used by \vec.  Instead of a little arrow
% over the letter, this creates a bold character.
\renewcommand{\vec}{\bm}

% Define unit vector notation.  If you don't override the
% behavior of \vec, you probably want to use the second one.
\newcommand{\unit}[1]{\hat{\bm{#1}}}
% \newcommand{\unit}[1]{\hat{#1}}

% Use this to refer to a single component of a unit vector.
\newcommand{\scalarunit}[1]{\hat{#1}}

% \toprule, \midrule, \bottomrule for tables
\usepackage{booktabs}

% \llbracket, \rrbracket
\usepackage{stmaryrd}

\usepackage{hyperref}
\hypersetup{
    colorlinks,
    citecolor=black,
    filecolor=black,
    linkcolor=black,
    urlcolor=black
}

% Compress lists of citations like [33,34,35,36,37] to [33-37]
\usepackage{cite}

% If you want to relax some of the SAND98-0730 requirements, use the "relax"
% option. It adds spaces and boldface in the table of contents, and does not
% force the page layout sizes.
% e.g. \documentclass[relax,12pt]{SANDreport}
%
% You can also use the "strict" option, which applies even more of the
% SAND98-0730 guidelines. It gets rid of section numbers which are often
% useful; e.g. \documentclass[strict]{SANDreport}

% The INLreport class uses \flushbottom formatting by default (since
% it's intended to be two-sided document).  \flushbottom causes
% additional space to be inserted both before and after paragraphs so
% that no matter how much text is actually available, it fills up the
% page from top to bottom.  My feeling is that \raggedbottom looks much
% better, primarily because most people will view the report
% electronically and not in a two-sided printed format where some argue
% \raggedbottom looks worse.  If we really want to have the original
% behavior, we can comment out this line...
\raggedbottom
\setcounter{secnumdepth}{5} % show 5 levels of subsection
\setcounter{tocdepth}{5} % include 5 levels of subsection in table of contents

% ---------------------------------------------------------------------------- %
%
% Set the title, author, and date
%
    \title{RAVEN User Manual}

    \author{%
      \\ \textit{The RAVEN team:}
       \\
      \\\ \underline{\textit{\textbf{Principal Investigator}}:}
       \\
      \\  \textbf{Cristian Rabiti}, \textit{the Boss}
      \\
      \\\underline{\textit{\textbf{Developers}}:}
       \\
      \\\textbf{Andrea Alfonsi},   \textit{the funny guy}
      \\\textbf{Joshua Cogliati},   \textit{the computer ace}
      \\\textbf{Diego Mandelli},   \textit{born old}
      \\\textbf{Robert Kinoshita}, \textit{he is not Japanese}
    }

    % There is a "Printed" date on the title page of a SAND report, so
    % the generic \date should [WorkingDir:]generally be empty.
    \date{}


% ---------------------------------------------------------------------------- %
% Set some things we need for SAND reports. These are mandatory
%
%\SANDnum{INL/EXT-14-xxxxx}
%\SANDprintDate{August 2014}
%\SANDauthor{ }
%\SANDreleaseType{Draft Release}

% ---------------------------------------------------------------------------- %
% Include the markings required for your SAND report. The default is "Unlimited
% Release". You may have to edit the file included here, or create your own
% (see the examples provided).
%
% \include{MarkOUO} % Not needed for unlimted release reports

\def\component#1{\texttt{#1}}

% ---------------------------------------------------------------------------- %
\newcommand{\systemtau}{\tensor{\tau}_{\!\text{SUPG}}}

% ---------------------------------------------------------------------------- %
%
% Start the document
%
\begin{document}
    \maketitle[Title]

    % ------------------------------------------------------------------------ %
    % An Abstract is required for SAND reports
    %
%    \begin{abstract}
%    \input abstract
%    \end{abstract}


    % ------------------------------------------------------------------------ %
    % An Acknowledgement section is optional but important, if someone made
    % contributions or helped beyond the normal part of a work assignment.
    % Use \section* since we don't want it in the table of context
    %
%    \clearpage
%    \section*{Acknowledgment}
%	Thanks to Ron Weasly for valuable discussions and helping
%	us finding new uses of magic.
%
%	The format of this report is based on information found
%	in~\cite{Sand98-0730}.


    % ------------------------------------------------------------------------ %
    % The table of contents and list of figures and tables
    % Comment out \listoffigures and \listoftables if there are no
    % figures or tables. Make sure this starts on an odd numbered page
    %
    \cleardoublepage		% TOC needs to start on an odd page
    \tableofcontents
    \listoffigures
    \listoftables


    % ---------------------------------------------------------------------- %
    % An optional preface or Foreword
%    \clearpage
%    \section*{Preface}
%    \addcontentsline{toc}{section}{Preface}
%	Although muggles usually have only limited experience with
%	magic, and many even dispute its existence, it is worthwhile
%	to be open minded and explore the possibilities.


    % ---------------------------------------------------------------------- %
    % An optional executive summary
    \clearpage
    \section*{Summary}
    \addcontentsline{toc}{section}{Summary}
    \input{Summary.tex}
%	Once a certain level of mistrust and skepticism has
%	been overcome, magic finds many uses in todays science
%	and engineering. In this report we explain some of the
%	fundamental spells and instruments of magic and wizardry. We
%	then conclude with a few examples on how they can be used
%	in daily activities at national Laboratories.


    % ---------------------------------------------------------------------- %
    % An optional glossary. We don't want it to be numbered
%    \clearpage
%    \section*{Nomenclature}
%    \addcontentsline{toc}{section}{Nomenclature}
%    \begin{description}
%	   \item[alohomoral]
%	    spell to open locked doors and containers
%	   \item[leviosa]
%	    spell to levitate objects
%    \item[remembrall]
%	    device to alert you that you have forgotten something
%    \item[wand]
%	    device to execute spells
%    \end{description}


    % ---------------------------------------------------------------------- %
    % This is where the body of the report begins; usually with an Introduction
    %
    %\SANDmain		% Start the main part of the report

\section{Introduction}
% High-level HYBRID description
One of the goals of the HYBRID modeling and simulation project is to assess the economic viability of hybrid systems in a market that contains renewable energy sources like wind. The hybrid system would be a nuclear reactor that not only generates electricity, but also provides heat to another plant that produces by-products, like hydrogen or desalinated water. The idea is that the possibility of selling heat to a heat user absorbs (at least part of) the market volatility introduced by the renewable energy sources.

The system that is studied is modular and made of an assembly of components. For example, a system could contain a hybrid nuclear reactor, a gas turbine, a battery and some renewables. This system would correspond to the size of a balance area, but in theory any size of system is imaginable. The system is modeled in the ‘Modelica/Dymola’ language.
To assess the economics of the system, an optimization procedure is varying different parameters of the system and tries to find the minimal cost of electricity production.

\subsection{Modelica Models}
Idaho National Laboratory (INL) has been developing the NHES package, a library of high-fidelity process models in the commercial Modelica language platform Dymola since early 2013 \cite{2017Report}, \cite{2016HTSE}, \cite{2018ThermalStorage}, \cite{2019NuScaleM4}. The Modelica language is a non-proprietary, object oriented, equation-based language that is used to conveniently model complex, physical systems. Modelica is an inherently time-dependent modeling language that allows the swift interconnection of independently developed models. Being an equation-based modeling language that employs differential algebraic equation (DAE) solvers, users can focus on the physics of the problem rather than the solving technique used, allowing faster model generation and ultimately analysis. This feature alongside system flexibility has led to the widespread use of the Modelica language across industry for commercial applications. System interconnectivity and the ability to quickly develop novel control strategies while still encompassing overall system physics is why INL has chosen to develop the Integrated Energy Systems (IES) framework in the Modelica language.

\subsection{Individual Components}
The current version of the NHES library employs both third party components from the Modelica Standard Library \cite{ModelicaAssociation} and TRANSFORM \cite{TRANSFORM} and components developed internal to the project for specific subsystems. For example, the NHES library contains a large variety of models for the development of a high-temperature steam electrolysis plant, a gas turbine, a basic Rankine cycle balance of plant, and a light water nuclear reactor. Components included in the library that support the development of these systems include 1-D pipes, pressurizers, condensers, turbines (steam and gas), heat exchangers, a simple logic-based battery, a nuclear fuel subchannel, etc. Third party models include numerous additional models including source/sink components (e.g., fluid boundary conditions), additional heat exchanger models, logical components for control system development, multi-body components, additional supporting functions (e.g., LAPACK, interpolation, smoothing), etc. Please see the specific libraries for additional information. 

\subsection{Hybrid Requirements}


The repository itself can be found here: \url{https://github.com/idaholab/hybrid}
 
\textbf{Software requirements are as follows:}

\begin{enumerate}
\item Commercial Modelica platform Dymola -- \url{https://www.3ds.com/products-services/catia/products/dymola/latest-release/}.
\item Risk Analysis and Virtual ENviroment (RAVEN) -- \url{https://raven.inl.gov/SitePages/Software%20Infrastructure.aspx}
\item Python 3 -- \url{https://docs.conda.io/en/latest/miniconda.html}
\item Microsoft Visual Studio Community Edition. -- \url{https://visualstudio.microsoft.com/downloads/}
\end{enumerate}

	

\textbf{Note}: Steps 3 and 4 can be accomplished by following the RAVEN installation instructions in step two. The installation procedure will be outlined below. 
All physical models are run within the Dymola simulation framework graphical user interface (GUI).  Background information on the Modelica as a language as well as good general guidance on coding practices can be found at the two references shown below. 
\begin{enumerate}
\item \url{https://webref.modelica.university/}
\item \url{https://mbe.modelica.university/}
\end{enumerate}


\input{ravenStructure.tex}
\input{runInfo.tex}
\input{ProbabilityDistributions.tex}
\input{sampler.tex}
\input{functions.tex}
\input{model.tex}
\input{step.tex}
\input{database_data.tex}
\input{OutStreamSystem.tex}
\input{existing_interfaces.tex}

    % ---------------------------------------------------------------------- %
    % References
    %
    \clearpage
    % If hyperref is included, then \phantomsection is already defined.
    % If not, we need to define it.
    \providecommand*{\phantomsection}{}
    \phantomsection
    \addcontentsline{toc}{section}{References}
    \bibliographystyle{ieeetr}
    \bibliography{raven_user_manual.bib}


    % ---------------------------------------------------------------------- %
    %

    % \printindex

    %\include{distribution}

\end{document}
