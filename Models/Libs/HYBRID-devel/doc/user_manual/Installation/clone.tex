\subsection{Cloning the Hybrid Repository}
\label{sec:clone raven}

The first step in installing the package is to clone the HYBRID repository. To do this, use
\begin{lstlisting}[language=bash]
git clone https://github.com/idaholab/HYBRID.git
\end{lstlisting}
This will download the repository into a folder called 'hybrid'. To go inside the folder, use
\begin{lstlisting}[language=bash]
cd hybrid
\end{lstlisting}


\subsubsection{Install RAVEN and its plugins as a sub-module}

The next step is to download and install RAVEN and the submodule (e.g. TEAL, HERON) plugins as a sub-module of the HYBRID repository. 

A submodule allows you to keep another Git repository in a subdirectory of your repository. The other repository has its own history, which does not interfere with the history of the current repository. This can be used to have external dependencies such as third party libraries for example.

In order to get RAVEN do the following in the hybrid folder

\begin{lstlisting}[language=bash]
git checkout devel
\end{lstlisting}

Update the Branch

\begin{lstlisting}[language=bash]
git pull
\end{lstlisting}

to add RAVEN as a submodule
\begin{lstlisting}[language=bash]
git submodule update --init --recursive
\end{lstlisting}

\textbf{Install and Compile RAVEN. }
Once you have downloaded RAVEN as a sub-module, you have to install it. go to the \href{https://github.com/idaholab/raven/wiki/intallationMain}{RAVEN Wiki} for information about how to install it. Run all the tests outlined in the RAVEN wiki. 

\subsubsection{Inform the Framework Paths}

In order to set up the hybrid repository, you must inform the framework about the location of the Dymola python interface. For doing so, navigate to the hybrid directory:

to add RAVEN as a submodule
\begin{lstlisting}[language=bash]
cd <path to your hybrid repository>/hybrid
\end{lstlisting}
Run the following command:
\begin{lstlisting}[language=bash]
./scripts/write_hybridrc.py -p DYMOLA_PATH
\end{lstlisting}

Where DYMOLAPATH is the path to the python interface egg folder in the DYMOLA installation locally. For example:
 
\begin{lstlisting}[language=bash]
./scripts/write_hybridrc.py -p 
	"/c/Program\ Files/Dymola\ 2020x/Modelica/Library/
	python_interface/dymola.egg"
\end{lstlisting}
